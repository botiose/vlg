\documentclass[11pt]{article}

% Use the following to compile
% mkdir tmp
% pdflatex -aux-directory=tmp -output-directory=tmp --shell-escape notes.tex

% Package use definitions
\usepackage[margin=1in]{geometry}
\usepackage{fancyhdr}
\usepackage[parfill]{parskip}
\usepackage{graphicx}
\usepackage{comment}
\usepackage[outputdir=tmp]{minted}
\usepackage[dvipsnames]{xcolor}
\usepackage{listings}
\usepackage[hidelinks]{hyperref}
\usepackage{amsmath}
\usepackage{amsfonts}
\usepackage{amssymb}
\usepackage{tcolorbox}
\usepackage{tabu}
\usepackage{upgreek}
\usepackage[ruled,vlined]{algorithm2e}
\usepackage[nottoc]{tocbibind}
\usepackage{natbib}

\setlength{\parindent}{11pt}
\setlength{\parskip}{0pt}

% Header and footer setup
\pagestyle{fancy} \rhead{\today} \lhead{Reinforcement Learning conceptions for
  Poker} \renewcommand{\headrulewidth}{1pt} \renewcommand{\footrulewidth}{1pt}

% Image directory specification
\graphicspath{ {./images/} }

% Settings minted option for the entire document
\definecolor{LightGray}{rgb}{0.9, 0.9, 0.9}
\setminted{frame=lines,framesep=2mm,bgcolor=LightGray,linenos,
  fontsize=\footnotesize, baselinestretch=1.2}

% Start of document
\begin{document}

% Title page and table of contents setup
\begin{titlepage}
  \begin{center}
    \vspace*{1cm} \Huge \textbf{Bounding Eccentricities}\\
    \vspace*{1\baselineskip} Very Large Graphs\\
    \vspace*{2\baselineskip} \large
    \begin{abstract}
    This paper is a summary of our research and development
    of the Bounding Eccentricities algorithm introduced in the 2013 article
    Computing the Eccentricity Distribution of Large Graphs by Frank W. Takes
    and Walter A. Kosters. In this report we restate all of the methodologies
    from the original paper that were applied during the implementation of the
    Bounding Eccentricities algorithm, as well as any other external concepts
    originating from other research on the same topic. We also show the results
    of various experiments using the same performance measures as in the
    original paper for the sake of simplifying comparison. Additionally, we also
    show the relative improvement brought forth by of all of the main
    methodologies introduced in the paper over previous versions, namely
    selection strategies and graph prunning.
    \end{abstract}
    \vfill \normalsize \textbf{Théo Minary : theo.minary@epita.fr}\\ \textbf{Jose
      A. Henriquez Roa : jose.henriquez-roa@epita.fr}\\
    \vspace*{2\baselineskip} \today \rhead{\today}
    \newpage
    \normalsize \tableofcontents
    \newpage
  \end{center}
\end{titlepage}
\section{Experiments}
In this section we present some performance analysis of our implementation of
this algorithm. To assess the performance of our implementation with respect to
the results found in section 6 of the paper TODO we provide some of the same
performance analysis. We have thus measured the number of execution of our
shortest path search algorithm required to compute all of the eccentricities on
the pruned graph.
\section{References}
\end{document}
