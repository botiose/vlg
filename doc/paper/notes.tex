\documentclass[11pt]{article}

% Use the following to compile
% mkdir tmp
% pdflatex -aux-directory=tmp -output-directory=tmp --shell-escape notes.tex

% Package use definitions
\usepackage[margin=1in]{geometry}
\usepackage{fancyhdr}
\usepackage[parfill]{parskip}
\usepackage{graphicx}
\usepackage{comment}
\usepackage[outputdir=tmp]{minted}
\usepackage[dvipsnames]{xcolor}
\usepackage{listings}
\usepackage[hidelinks]{hyperref}
\usepackage{amsmath}
\usepackage{amsfonts}
\usepackage{amssymb}
\usepackage{tcolorbox}
\usepackage{tabu}
\usepackage{upgreek}
\usepackage[ruled,vlined]{algorithm2e}
\usepackage[nottoc]{tocbibind}
\usepackage{natbib}

\setlength{\parindent}{11pt}
\setlength{\parskip}{0pt}

% Header and footer setup
\pagestyle{fancy} \rhead{\today} \lhead{Reinforcement Learning conceptions for
  Poker} \renewcommand{\headrulewidth}{1pt} \renewcommand{\footrulewidth}{1pt}

% Image directory specification
\graphicspath{ {./images/} }

% Settings minted option for the entire document
\definecolor{LightGray}{rgb}{0.9, 0.9, 0.9}
\setminted{frame=lines,framesep=2mm,bgcolor=LightGray,linenos,
  fontsize=\footnotesize, baselinestretch=1.2}

% Start of document
\begin{document}

% Title page and table of contents setup
\begin{titlepage}
  \begin{center}
    \vspace*{1cm} \Huge \textbf{Bounding Eccentricities}\\
    \vspace*{1\baselineskip} Very Large Graphs\\
    \vspace*{2\baselineskip} \large
    \begin{abstract}
      \noindent
    This paper is a summary of our research and development
    of the Bounding Eccentricities algorithm introduced in the 2013 article
    Computing the Eccentricity Distribution of Large Graphs by Frank W. Takes
    and Walter A. Kosters. In this report we restate all of the methodologies
    from the original paper that were applied during the implementation of the
    Bounding Eccentricities algorithm, as well as any other external concepts
    originating from other research on the same topic. We also show the results
    of various experiments using the same performance measures as in the
    original paper for the sake of simplifying comparison. Additionally, we also
    show the relative improvement brought forth by of all of the main
    methodologies introduced in the paper over previous versions, namely
    selection strategies and graph prunning.
    \end{abstract}
    \vfill \normalsize \textbf{Théo Minary : theo.minary@epita.fr}\\ \textbf{Jose
      A. Henriquez Roa : jose.henriquez-roa@epita.fr}\\
    \vspace*{2\baselineskip} \today \rhead{\today}
    \newpage
    \normalsize \tableofcontents
    \newpage
  \end{center}
\end{titlepage}
\section{Experiments}
In this section we present some performance analysis of our implementation of
this algorithm. To assess the performance of our implementation with respect to
the results found in section 6 of the paper [1] we provide some of the same
performance measures. We have thus measured the number of execution of our
shortest path search algorithm required to compute all of the eccentricities. In
this section we shall first analyse the performance of the different selection
strategies as well as the graph prunning strategy given in the paper. The number
of iteration taken after the implementation of each method presented in the
respective columns for the ca-CondMat, ca-HepPh and ca-HepTh is given in the
following table:\\
\begin{center}
 \begin{tabular}{||c c c c c||} 
   \hline
   Dataset & Random & Pruning & Interchanging bound & Degree centrality \\
   \hline\hline
   ca-CondMat & 9439 & 9268 & 3486 & 3271 \\ 
   \hline
   ca-HepPh & 8444 & 8211 & 1662 & 1589 \\
   \hline
   ca-HepTh & 6175 & 5886 & 1101 & 1053 \\
   \hline
 \end{tabular}
\end{center}
\medskip
\noindent
The best performing variant was evaluated on 3 more datasets of higher
sizes. The following table shows the results:\\
\begin{tcolorbox}
  \begin{center}
    TODO
  \end{center}
\end{tcolorbox}
\section{Conclusion}
We have thus implemented the algorithm and have found very similar results for
each of the graphs used in the paper. We also have seen the relative improvement
of each optimization method proposed in the paper. However, there are still some
improvements that can be performed. A possible route for improvement would be to
find a better candidate selection method, however this would require more
research to be performed. Another route for improvement that would require less
effort would be to parallelise the shortest path computations to leverage as
most as possible multiple core systems which are much more popular today than
they were in 2013.
\section{References}
\begin{enumerate}
\item Takes, F.W.; Kosters, W.A. Computing the Eccentricity Distribution of Large Graphs. Algorithms 2013, 6, 100-118. https://doi.org/10.3390/a6010100
\item Frank W. Takes and Walter A. Kosters. 2011. Determining the diameter of small world networks. In Proceedings of the 20th ACM international conference on Information and knowledge management (CIKM '11). Association for Computing Machinery, New York, NY, USA, 1191–1196. DOI:https://doi.org/10.1145/2063576.2063748
\end{enumerate}
\end{document}
